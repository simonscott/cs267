% !TEX TS-program = pdflatex
% !TEX encoding = UTF-8 Unicode

% This is a simple template for a LaTeX document using the "article" class.
% See "book", "report", "letter" for other types of document.

\documentclass[11pt]{article} % use larger type; default would be 10pt

\usepackage[utf8]{inputenc} % set input encoding (not needed with XeLaTeX)

%%% Examples of Article customizations
% These packages are optional, depending whether you want the features they provide.
% See the LaTeX Companion or other references for full information.

%%% PAGE DIMENSIONS
\usepackage{geometry} % to change the page dimensions
\geometry{letterpaper} % or letterpaper (US) or a5paper or....
% \geometry{margins=2in} % for example, change the margins to 2 inches all round
% \geometry{landscape} % set up the page for landscape
%   read geometry.pdf for detailed page layout information

\usepackage{graphicx} % support the \includegraphics command and options

% \usepackage[parfill]{parskip} % Activate to begin paragraphs with an empty line rather than an indent

%%% PACKAGES
\usepackage{booktabs} % for much better looking tables
\usepackage{array} % for better arrays (eg matrices) in maths
\usepackage{paralist} % very flexible & customisable lists (eg. enumerate/itemize, etc.)
\usepackage{verbatim} % adds environment for commenting out blocks of text & for better verbatim
\usepackage{subfig} % make it possible to include more than one captioned figure/table in a single float
\usepackage{amsmath}
\usepackage{amssymb}
% These packages are all incorporated in the memoir class to one degree or another...

%%% HEADERS & FOOTERS
\usepackage{fancyhdr} % This should be set AFTER setting up the page geometry
\pagestyle{fancy} % options: empty , plain , fancy
\renewcommand{\headrulewidth}{0pt} % customise the layout...
\lhead{}\chead{}\rhead{}
\lfoot{}\cfoot{\thepage}\rfoot{}

%%% SECTION TITLE APPEARANCE
\usepackage{sectsty}
\allsectionsfont{\sffamily\mdseries\upshape} % (See the fntguide.pdf for font help)
% (This matches ConTeXt defaults)

%%% ToC (table of contents) APPEARANCE
\usepackage[nottoc,notlof,notlot]{tocbibind} % Put the bibliography in the ToC
\usepackage[titles,subfigure]{tocloft} % Alter the style of the Table of Contents
\renewcommand{\cftsecfont}{\rmfamily\mdseries\upshape}
\renewcommand{\cftsecpagefont}{\rmfamily\mdseries\upshape} % No bold!

%%% END Article customizations

%%% The "real" document content comes below...

\title{CS267 Assignment 3}
\author{Patrick Li, Simon Scott}
%\date{} % Activate to display a given date or no date (if empty),
         % otherwise the current date is printed 

\begin{document}
\maketitle
\parskip 7.2pt

\section{Introduction and Aims}

The aim of this assignment was ...

\section{Method of Parallelization}

How we parallelized the code. We blocked the columns, with each processor having single set of adjacent columns. Synchronize after each row.

\section{Optimization Techniques}

\subsection{Only Store Two Rows}

Saves memory and makes accessing memory easier.

\subsection{Elimination of Backtracking}

Count array to get total count.
Searching for change in value to get weight.

\subsection{Bulk Memory Copies}

memget, rather than reading each element separately.

\subsection{Synchronizing via Spin-Locks}

Use progress array as spin-lock to avoid costly upc\_barrier operation. Lock must be declared volatile.

\section{Results}

\subsection{Strong Scaling}

How does time vary with number of processors, for fixed global problem size. Show figure 1, and explain.

\subsection{Weak Scaling}

How does solution time vary with number of processors, for fixed per-processor problem size. Show figure 3, and explain.

\subsection{Time versus Problem Size}

How does solution time vary for fixed number of processors, when problem sized increased. Shows figure 2, and explain.

\subsection{Effect of Optimizations on Performance}

\subsubsection{Bulk Memory Copies}

Give \% speed increase for a single P, C, N combination.

\subsubsection{Synchronizing via Spin-Locks}

Give \% speed increase for a single P, C, N combination.

\section{Conclusion}

TODO

\end{document}
